% This and that
%------------------
\usepackage{etex} % use the etex engine, more variables etc.
\usepackage{etoolbox}
\usepackage[utf8]{inputenc}
\usepackage{cmap}    % Correct encoding for Umlaute in pdf
\usepackage[T1]{fontenc}    % Correct hyphenation for non-ASCII characters
\usepackage[english,french]{babel}
\usepackage{calc}
\usepackage[left]{eurosym}
\usepackage{thumbpdf}
\usepackage{cite} % Order citations automatically (such as [1-4,5,8])
% \usepackage{multibib}
\usepackage{array} % Extend array and tabular with column formatting
\usepackage{hyperref}
\hypersetup{final,
	        plainpages=false,
	        pdfpagelabels,
	        pagebackref,
	        %hyperfootnotes=false,
	        colorlinks=true,
	        linkcolor=blue,
	        urlcolor=black,
	        citecolor=blue,
	        pdfpagemode=UseOutlines,
	        pdfstartview=FitH,
	        pdfborder={0 0 0}}
\usepackage{url}
\usepackage[per-mode=symbol,bracket-unit-denominator=false,detect-all,load-configurations=binary,binary-units]{siunitx}
\DeclareSIUnit\loc{LOC}
\DeclareSIUnit\permil{\textperthousand}
\usepackage{pifont}  % Symbol characters
\usepackage[weather]{ifsym}
\usepackage{wasysym}
\usepackage{varwidth}

\providecommand{\todo}[1]{{\Large TODO: #1}}
\providecommand{\newterm}[1]{\emph{#1}}
\providecommand{\indexed}[1]{\index{#1}#1}
\providecommand{\product}[1]{{\scshape #1}\index{#1@\textsc{#1}}}
\newcommand{\HRule}{\rule{\linewidth}{0.5mm}}
\newcommand{\ie}{i.\,e.\ }
\newcommand{\eg}{e.\,g.\ }
\newcommand{\cf}{cf.\ }
\newcommand{\dotfillbox}[1]{\makebox[#1]{\dotfill}}


% Tables
%-------------
\usepackage{tabularx}
% Split a cell into multiple rows/columns
\usepackage{multicol}
\usepackage{multirow}
% \usepackage{slashbox} % Diagonal cells
\usepackage{booktabs}
\usepackage{rotating}


% Graphics
%--------------
\usepackage[all]{xy}
\usepackage{xcolor}
\usepackage{graphicx}
\usepackage{tikz}
\usetikzlibrary{arrows,positioning,shapes,topaths,calc,fit,backgrounds,matrix,shadows,automata,patterns,decorations.pathmorphing,decorations.pathreplacing,decorations.text,circuits.logic.US,trees,mindmap,spy,lindenmayersystems}
\makeatletter
\newcommand{\gettikzxy}[3]{%
  \tikz@scan@one@point\pgfutil@firstofone#1\relax
  \edef#2{\the\pgf@x}%
  \edef#3{\the\pgf@y}%
}
\makeatother
\usepackage{gnuplot-lua-tikz} % GNUplot plots
% \usepackage{pgfplots}
\usepackage{chronology} % timeline
\usepackage{fancybox}
\newcommand{\MYhref}[3][black]{\href{#2}{\color{#1}{#3}}}  % Enforce black hyperlinks
\definecolor{myyellow}{RGB}{242,226,149}
\definecolor{darkgreen}{RGB}{0,128,0}
\definecolor{beamerbackgroundcolor}{RGB}{250,250,250}


% Math
%--------
\usepackage{amsmath,amssymb,amstext,amsthm,amsfonts}
\usepackage{dsfont} %Font for nice set symbols (R, N, Z, ...)
\usepackage{nicefrac} % "Marktfrauenbruch"
% Set symbols
\newcommand{\R}{\ensuremath{\mathds{R}}}
\renewcommand{\P}{\ensuremath{\mathds{P}}}
\newcommand{\N}{\ensuremath{\mathds{N}}}
\newcommand{\Z}{\ensuremath{\mathds{Z}}}
\newcommand{\Q}{\ensuremath{\mathds{Q}}}
\newcommand{\F}{\ensuremath{\mathds{F}}}
\newcommand{\C}{\ensuremath{\mathds{C}}}
% Operators
\providecommand{\abs}[1]{\left\lvert #1 \right\rvert}
\providecommand{\norm}[1]{\left\lVert #1 \right\rVert}
\providecommand{\floor}[1]{\left\lfloor #1 \right\rfloor}
\providecommand{\ceil}[1]{\left\lceil #1 \right\rceil}
\providecommand{\svert}{\; \vert \;} %Single vertical bar
\DeclareMathOperator{\grad}{grad} % Gradient
\DeclareMathOperator{\rot}{rot} % Rotation
\DeclareMathOperator{\rg}{rg}
\DeclareMathOperator{\deF}{def}
\DeclareMathOperator{\ran}{ran}
\DeclareMathOperator{\dist}{d}
\DeclareMathOperator{\rx}{rx}
\DeclareMathOperator{\tx}{tx}
\DeclareMathOperator{\req}{req}
\DeclareMathOperator{\avg}{avg}
\DeclareMathOperator{\vol}{vol}
\DeclareMathOperator{\identical}{id}
\providecommand{\id}{\ensuremath{\textsf{id}}}
\DeclareMathOperator{\prob}{\mathbf{P}}
\DeclareMathOperator{\E}{\mathbf{E}}
\DeclareMathOperator{\percentile}{P}
\DeclareMathOperator{\definedas}{\mathrel{\mathop:}=}
\DeclareMathOperator{\asdefined}{\mathrel=:}
% Arbitrarily long double arrows
\makeatletter
\def\Relbar{\mathrel{\smash=}}
\def\Leftarrowfill@{\arrowfill@\Leftarrow\Relbar\Relbar}
\def\Rightarrowfill@{\arrowfill@\Relbar\Relbar\Rightarrow}
\newcommand{\xRightarrow}[2][]{\ext@arrow 0359\Rightarrowfill@{#1}{#2}}
\newcommand{\xLeftarrow}[2][]{\ext@arrow 3095\Leftarrowfill@{#1}{#2}}
\makeatother


% Computer Science
%--------------------------
\usepackage[ruled, vlined, algo2e]{algorithm2e}

\usepackage[final]{listings}
\lstset{numbers=left, numberstyle=\tiny, stepnumber=2, numbersep=5pt, frame=lines, basicstyle=\small, language=c}

% \usepackage{tikz-er2} % ER-Diagrams with TikZ
\usepackage[underline=false,rounded corners=false]{pgf-umlsd} % UML diagrams with TikZ
% \usepackage{rail}
% \railalias{lbrace}{\{}
% \railalias{rbrace}{\}}
% \railalias{underscore}{\_}
% \railalias{dollar}{\$}
% \railalias{percent}{\%}
% \railalias{ampersand}{<>}
% \railalias{backslash}{\char"5C}
% \railalias{tilde}{$\sim$}
% \railalias{ampersand}{\&}
% \railalias{doubleampersand}{\&\&}
% \railalias{etc}{$\cdots$}
% \railalias{xorhat}{$\wedge$}
% \railterm{lbrace,rbrace,dollar,percent,ampersand,backslash,underscore,tilde,ampersand,doubleampersand,xorhat}
% \usepackage{qtree}
\newcommand{\np}{\ensuremath{\mathcal{NP}}}
\renewcommand{\O}{\ensuremath{\mathcal{O}}}


% Typography
%------------------
\usepackage{microtype}
\usepackage{slantsc} % Combine sc fonts with italic, bold, ...
\clubpenalty = 9000 % No Schusterjungen
\widowpenalty = 9000 \displaywidowpenalty = 9000 % No Hurenkinder
% Roman numbers
\newcommand{\romannum}[1]{
	\ifnum#1<1
		\ifnum#1=0
			o
		\else
			-\romannumeral -#1
		\fi
	\else
		\romannumeral #1
	\fi}
\DeclareRobustCommand{\Romannum}[1]{\MakeUppercase{\romannum{#1}}}


% Sankey diagrams
%------------------------
\newif\ifsankeydebug

\newenvironment{sankeydiagram}[1][]{

  \def\sankeyflow##1##2{% sn, en
    \path[sankey fill]
    let
    \p1=(##1.north east),\p2=(##1.south east),
    \n1={atan2(\x1-\x2,\y1-\y2)-90},
    \p3=(##2.north west),\p4=(##2.south west),
    \n2={atan2(\x3-\x4,\y3-\y4)+90}
    in
    (\p1) to[out=\n1,in=\n2] (\p3) --
    (\p4) to[in=\n1,out=\n2] (\p2) -- cycle;
    \draw[sankey draw]
    let
    \p1=(##1.north east),\p2=(##1.south east),
    \n1={atan2(\x1-\x2,\y1-\y2)-90},
    \p3=(##2.north west),\p4=(##2.south west),
    \n2={atan2(\x3-\x4,\y3-\y4)+90}
    in
    (\p1) to[out=\n1,in=\n2] (\p3)
    (\p4) to[in=\n1,out=\n2] (\p2);
  }


  \tikzset{
    sankey tot length/.store in=\sankeytotallen,
    sankey tot quantity/.store in=\sankeytotalqty,
    sankey min radius/.store in=\sankeyminradius,
    sankey arrow length/.store in=\sankeyarrowlen,
    sankey debug/.is if=sankeydebug,
    sankey debug=false,
    sankey flow/.style={
      to path={
        \pgfextra{
          \pgfinterruptpath
          \edef\sankeystart{\tikztostart}
          \edef\sankeytarget{\tikztotarget}
          \sankeyflow{\sankeystart}{\sankeytarget}
          \endpgfinterruptpath
        }
      },
    },
    sankey node/.style={
      inner sep=0,minimum height={sankeyqtytolen(##1)},
      minimum width=0,draw=none,line width=0pt,
    },
    % sankey angle
    sankey angle/.store in=\sankeyangle,
    % sankey default styles
    sankey fill/.style={line width=0pt,fill,white},
    sankey draw/.style={draw=black,line width=.4pt},
  }

  \newcommand\sankeynode[4]{%prop,orientation,name,pos
    \node[sankey node=##1,rotate=##2] (##3) at (##4) {};
    \ifsankeydebug
    \begin{pgfonlayer}{sankeydebug}
      \draw[red,|-|] (##3.north west) -- (##3.south west);
      \pgfmathsetmacro{\len}{sankeyqtytolen(##1)/3}
      \draw[red] (##3.west)
      -- ($(##3.west)!\len pt!90:(##3.south west)$)
      node[font=\tiny,text=black] {##3};
    \end{pgfonlayer}
    \fi
  }

  \newcommand\sankeynodestart[4]{%prop,orientation,name,pos
    \sankeynode{##1}{##2}{##3}{##4}
    \begin{scope}[shift={(##3)},rotate=##2]
      \path[sankey fill]
      (##3.north west) -- ++(-\sankeyarrowlen,0)
      -- ([xshift=-\sankeyarrowlen/6]##3.west)
      -- ([xshift=-\sankeyarrowlen]##3.south west)
      -- (##3.south west) -- cycle;
      \path[sankey draw]
      (##3.north west) -- ++(-\sankeyarrowlen,0)
      -- ([xshift=-\sankeyarrowlen/6]##3.west)
      -- ([xshift=-\sankeyarrowlen]##3.south west)
      -- (##3.south west);
    \end{scope}
  }

  \newcommand\sankeynodeend[4]{%prop,orientation,name,pos
    \sankeynode{##1}{##2}{##3}{##4}
    \begin{scope}[shift={(##3)},rotate=##2]
      \path[sankey fill]
      (##3.north east)
      -- ([xshift=\sankeyarrowlen]##3.east)
      -- (##3.south west) -- cycle;
      \path[sankey draw]
      (##3.north east)
      -- ([xshift=\sankeyarrowlen]##3.east)
      -- (##3.south west);
    \end{scope}
  }

  \newcommand\sankeyadvance[3][]{%newname,name,distance
    \edef\name{##2}
    \ifstrempty{##1}{
      \def\newname{##2}
      \edef\name{##2-old}
      \path [late options={name=##2,alias=\name}];
    }{
      \def\newname{##1}
    }
    \path
    let
    % sankey node angle
    \p1=(##2.north east),
    \p2=(##2.south east),
    \n1={atan2(\x1-\x2,\y1-\y2)-90},
    % sankey prop
    \p3=($(\p1)-(\p2)$),
    \n2={sankeylentoqty(veclen(\x3,\y3))},
    % next position
    \p4=($(##2.east)!##3!-90:(##2.north east)$)
    in
    \pgfextra{
      \pgfmathsetmacro{\prop}{\n2}
      \pgfinterruptpath
      \sankeynode{\prop}{\n1}{\newname}{\p4}
      \path (\name) to[sankey flow] (\newname);
      \endpgfinterruptpath
    };
  }

  \newcommand\sankeyturn[3][]{%newname,name,angle
    \edef\name{##2}
    \ifstrempty{##1}{
      \def\newname{##2}
      \edef\name{##2-old}
      \path [late options={name=##2,alias=\name}];
    }{
      \def\newname{##1}
    }
    \ifnumgreater{##3}{0}{
      \typeout{turn acw: ##3}
      \path
      let
      % sankey node angle
      \p1=(##2.north east),
      \p2=(##2.south east),
      \p3=($(\p1)!-\sankeyminradius!(\p2)$),
      \n1={atan2(\x1-\x2,\y1-\y2)-90},
      % sankey prop
      \p4=($(\p1)-(\p2)$),
      \n2={sankeylentoqty(veclen(\x4,\y4))},
      \p5=(##2.east),
      \p6=($(\p3)!1!##3:(\p5)$)
      in
      \pgfextra{
        \pgfmathsetmacro{\prop}{\n2}
        \pgfinterruptpath
        % \fill[red] (\p3) circle (2pt);
        % \fill[blue](\p6) circle (2pt);
        \sankeynode{\prop}{\n1+##3}{\newname}{\p6}
        \path (\name) to[sankey flow] (\newname);
        \endpgfinterruptpath
      };
    }{
      \typeout{turn acw: ##3}
      \path
      let
      % sankey node angle
      \p1=(##2.south east),
      \p2=(##2.north east),
      \p3=($(\p1)!-\sankeyminradius!(\p2)$),
      \n1={atan2(\x1-\x2,\y1-\y2)+90},
      % sankey prop
      \p4=($(\p1)-(\p2)$),
      \n2={sankeylentoqty(veclen(\x4,\y4))},
      \p5=(##2.east),
      \p6=($(\p3)!1!##3:(\p5)$)
      in
      \pgfextra{
        \pgfmathsetmacro{\prop}{\n2}
        \pgfinterruptpath
        % \fill[red] (\p3) circle (2pt);
        % \fill[blue](\p6) circle (2pt);
        \sankeynode{\prop}{\n1+##3}{\newname}{\p6}
        \path (\name) to[sankey flow] (\newname);
        \endpgfinterruptpath
      };
    }
  }

  \newcommand\sankeyfork[2]{%name,list of forks
    \def\name{##1}
    \def\listofforks{##2}
    \xdef\sankeytot{0}
    \path
    let
    % sankey node angle
    \p1=(\name.north east),
    \p2=(\name.south east),
    \n1={atan2(\x1-\x2,\y1-\y2)-90},
    % sankey prop
    \p4=($(\p1)-(\p2)$),
    \n2={sankeylentoqty(veclen(\x4,\y4))}
    in
    \pgfextra{
      \pgfmathsetmacro{\iprop}{\n2}
    }
    \foreach \prop/\name[count=\c] in \listofforks {
      let
      \p{start \name}=($(\p1)!\sankeytot/\iprop!(\p2)$),
      \n{nexttot}={\sankeytot+\prop},
      \p{end \name}=($(\p1)!\n{nexttot}/\iprop!(\p2)$),
      \p{mid \name}=($(\p{start \name})!.5!(\p{end \name})$)
      in
      \pgfextra{
        \xdef\sankeytot{\n{nexttot}}
        \pgfinterruptpath
        \sankeynode{\prop}{\n1}{\name}{\p{mid \name}}
        \endpgfinterruptpath
      }
    }
    \pgfextra{
      \pgfmathsetmacro{\diff}{abs(\iprop-\sankeytot)}
      \pgfmathtruncatemacro{\finish}{\diff<0.01?1:0}
      \ifnumequal{\finish}{1}{}{
        \message{*** Warning: bad sankey fork (maybe)...}
        \message{\iprop-\sankeytot}
      }
    };
  }

  \tikzset{
    % default values,
    declare function={
      sankeyqtytolen(\qty)=\qty/\sankeytotalqty*\sankeytotallen;
      sankeylentoqty(\len)=\len/\sankeytotallen*\sankeytotalqty;
    },
    sankey tot length=100pt,
    sankey tot quantity=100,
    sankey min radius=30pt,%
    sankey arrow length=10pt,%
    % user values
    #1}
}{
}
